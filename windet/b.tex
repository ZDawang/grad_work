
\documentclass[11pt,UTF8]{ctexart}
\usepackage[top=2cm, bottom=2cm, left=2cm, right=2cm]{geometry}
\usepackage[ruled,linesnumbered,boxed]{algorithm2e}
\usepackage{algorithmicx}
\usepackage{algpseudocode}
\usepackage{amsmath}
\usepackage{amssymb}%空集符号

\renewcommand{\algorithmicrequire}{\textbf{输入:}}
\renewcommand{\algorithmicensure}{\textbf{输出:}}

\begin{document}
    \begin{algorithm}[H]
     \caption{卷积运算函数实现}
      \KwIn{卷积运算参数、输入特征图地址、输出特征图地址、卷积核参数地址、下一层存储参数}
      \KwOut{无}
        初始化参数及配置片外存储功能单元\;
        \For{$input_{depth}=0; input_{depth} \le input_{max\_depth};input_{depth}=input_{depth}+input_{para}$}
        {
            向片外存储功能单元发送读请求读取输入特征图数据\;
            将数据读入输入缓存功能单元中\;
            \For{$kernel_{depth}=0; kernel_{depth} \le kernel_{max\_depth};kernel_{depth}=kernel_{depth}+output_{para}$}
            {
                向片外存储功能单元发送读请求读取卷积核参数\;
                将卷积核参数从片外存储功能单元读入内存中\;
                \For{$kernel_{height}=0; kernel_{height} \le kernel_{max\_height};kernel_{height}=kernel_{height}+3$}
                {
                    \For{$kernel_{width}=0; kernel_{width} \le kernel_{max\_width};kernel_{width}=kernel_{width}+3$}
                    {
                        从内存中加载参数到卷积运算单元中\;
                        \For{$h=0; h \le input_{max\_height}-kernel_{max\_height}+kernel_{height};h=h+stride$}
                        {
                            \For{$w=0; h \le input_{max\_width}-kernel_{max\_width}+kernel_{width};w=w+stride$}
                            {
                                 从输入缓存功能单元中使用旋转存储结构读取数据,若为偏置层则全读为1\;
                                 将数据输入到卷积运算单元中计算\;
                                 将计算结果写入输出缓存功能单元中,更新模式选择初始化或累加\;
                            }
                        }
                    }
                }
            }
        }
        根据下一层存储参数将输出缓存中的计算结果写入片外存储中\;
        return\;
    \end{algorithm}







    \begin{algorithm}[H]
     \caption{池化运算函数实现}
      \KwIn{池化运算参数、输入特征图地址、输出特征图地址、下一层存储参数}
      \KwOut{无}
        初始化参数及配置片外存储功能单元\;
        \For{$input_{depth}=0; input_{depth} \le input_{max\_depth};input_{depth}=input_{depth}+input_{para}$}
        {
            向片外存储功能单元发送读请求读取输入特征图数据\;
            将数据从片外存储功能单元读入输入缓存功能单元中\;
            \For{$pool_{height}=0; pool_{height} \le pool_{max\_height};pool_{height}=pool_{height}+3$}
            {
                \For{$pool_{width}=0; pool_{width} \le pool_{max\_width};pool_{width}=pool_{width}+3$}
                {
                    \For{$h=0; h \le input_{max\_height}-pool_{max\_height}+pool_{height};h=h+stride$}
                    {
                        \For{$w=0; h \le input_{max\_width}-pool_{max\_width}+pool_{width};w=w+stride$}
                        {
                             从输入缓存功能单元中使用旋转存储结构读出数据\;
                             将数据输入到池化运算单元中计算,根据参数选择最大化或者平均池化\;
                             将计算结果写入输出缓存功能单元中,更新模式选择初始化或根据参数选择累加与最大化\;
                        }
                    }
                }
            }
        }
        根据下一层存储参数将输出缓存中的计算结果写入片外存储中\;
        return\;
    \end{algorithm}



    \begin{algorithm}[H]
     \caption{激活函数层函数实现}
      \KwIn{激活函数层参数、输入特征图地址、输出特征图地址、下一层存储参数}
      \KwOut{无}
        初始化参数及配置片外存储功能单元\;
        \For{$input_{depth}=0; input_{depth} \le input_{max\_depth};input_{depth}=input_{depth}+input_{para}$}
        {
            向片外存储功能单元发送读请求读取输入特征图数据\;
            \For{$l=0; l \le input_{max\_length};l=l+fc_{para}$}
            {
                从片外存储功能单元中读出输入特征图数据\;
                将输入特征图数据输入到激活函数运算单元中计算,根据参数选择激活函数\;
                将计算结果写入输出缓存功能单元中,更新模式选择初始化\;
            }
        }
        根据下一层存储参数将输出缓存中的计算结果写入片外存储中\;
        return\;
    \end{algorithm}
    
    
    
    \begin{algorithm}[H]
     \caption{批量归一化层函数实现}
      \KwIn{批量归一化层参数,输入特征图地址、输出特征图地址、参数地址、下一层存储参数}
      \KwOut{无}
        初始化参数及配置片外存储功能单元\;
        向片外存储功能单元发送读请求读取参数\;
        将参数从片外存储功能单元读入内存中\;
        \For{$input_{depth}=0; input_{depth} \le input_{max\_depth};input_{depth}=input_{depth}+input_{para}$}
        {
            向片外存储功能单元发送读请求读取输入特征图数据\;
            将数据从片外存储功能单元读入输入缓存功能单元中\;
            \For{$l=0; l \le input_{max\_length};l=l+fc_{para}$}
            {
                从输入缓存功能单元中使用顺序存储结构读出数据\;
                将数据输入到更新均值单元中
            }
            计算均值\;
            \For{$l=0; l \le input_{max\_length};l=l+fc_{para}$}
            {
                从输入缓存功能单元中使用顺序存储结构读出数据\;
                将数据输入到更新方差单元中
            }
            计算方差\;
            \For{$l=0; l \le input_{max\_length};l=l+fc_{para}$}
            {
                从输入缓存功能单元中使用顺序存储结构读出数据\;
                将数据输入到归一化与变换重构运算单元中进行计算\;
                将计算结果写入输出缓存功能单元中,更新模式选择初始化\;
            }
        }
        根据下一层存储参数将输出缓存中的计算结果写入片外存储中\;
        return\;
    \end{algorithm}


    \begin{algorithm}[H]
     \caption{全连接层函数实现}
      \KwIn{全连接层运算参数、输入特征图地址、输出特征图地址、参数地址、下一层存储参数}
      \KwOut{无}
        初始化参数及配置片外存储功能单元\;
        向片外存储功能单元发送读请求读取输入数据\;
        将输入数据从片外存储功能单元读入输入缓存中\;
        \For{$l=0; l \le input_{max\_length};l=l+1$}
        {
            从输入缓存功能单元中使用顺序存储结构读取数据\;
            向片外存储功能单元发送读请求读取参数\;
            \For{$d=0; d \le para_{max\_depth};d=d+fc_{para}$}
            {
                从片外存储功能单元中读出参数数据\;
                将参数数据与输入数据输入到全连接层运算单元中计算\;
                将计算结果写入输出缓存功能单元中,更新模式选择初始化或累加\;
            }
        }
        根据下一层存储参数将输出缓存中的计算结果写入片外存储中\;
        return\;
    \end{algorithm}

\end{document}

